\documentclass{article}
\usepackage[utf8]{inputenc}
\usepackage{hyperref}
\usepackage{listings}
\usepackage{color}

\definecolor{dkgreen}{rgb}{0,0.6,0}
\definecolor{gray}{rgb}{0.5,0.5,0.5}
\definecolor{mauve}{rgb}{0.58,0,0.82}

\lstdefinelanguage{JavaScript}{
  keywords={typeof, new, true, false, catch, function, return, null, catch, switch, var, if, in, while, do, else, case, break},
  keywordstyle=\color{blue}\bfseries,
  ndkeywords={class, export, boolean, throw, implements, import, this},
  ndkeywordstyle=\color{darkgray}\bfseries,
  identifierstyle=\color{black},
  sensitive=false,
  comment=[l]{//},
  morecomment=[s]{/*}{*/},
  commentstyle=\color{purple}\ttfamily,
  stringstyle=\color{red}\ttfamily,
  morestring=[b]',
  morestring=[b]"
}

\lstset{frame=tb,
  language=Javascript,
  aboveskip=3mm,
  belowskip=3mm,
  showstringspaces=false,
  columns=flexible,
  basicstyle={\small\ttfamily},
  numbers=none,
  numberstyle=\tiny\color{gray},
  keywordstyle=\color{blue},
  commentstyle=\color{dkgreen},
  stringstyle=\color{mauve},
  breaklines=true,
  breakatwhitespace=true,
  tabsize=3
}




\title{\textbf{Angular Training Notes}}
\author{Mounaim Zaryouhi \\ email: \href{mailto:m.zaryouhi@ancfcc.gov.ma}{m.zaryouhi@ancfcc.gov.ma}}
\date{20 February 2020}

\begin{document}

\maketitle

\section{Introduction}
\begin{itemize}
    \item 
    Typescript is a superset of Javascript
    \item
    Compiles to Javascript
    \item
    Comes with types : number,string,boolean...etc.
    \item Angular enables \textbf{microservices} architecture on the front end part
    \item Microservices allow Front End and Back End decoupling, with communication through APIs
    \item AngularJS is only for version 1
    \item from version 2, it's called only \textbf{Angular}
    \item there is no version 3, Angular 2 to Angular 4 directly
    \item Angular enables Single Page Applications
    \item La structure d'un URL : 
    \begin{verbatim}
        URL = URI + Segment (Static or Dynamic) + queries
       \\ Example : https://www.google.com/search/ma?q=---------&s=-----------
       URI : www.google.com
       search: static
       ma : dynamic
       q,s : queries
    \end{verbatim}
    
    
    
\end{itemize}


\section{Setting up Environment}
\begin{itemize}
    \item Install Node JS
    \item Install Visual Studio Code
    \item check node is installed : Run command \textit{node -v}
    \item check npm is installed : Run command \textit{npm -v}
    \item install Agular :  \textit{npm install -g angular/cli} 
    \item check Angular is installed : \textit{ng version} 
    \item -g option makes npm install packages globally
    \item generate new project
    \\ \textbf{Example : } Let's create a project called ancfcc-basic :
    
     run this command : \textit{ng new ancfcc-basic}
    \item Go to Visual Studio Code and open the folder ancfcc-basic
    \item Go to ancfcc-basic folder in the command line and launch the app ancfcc-basic with : \textit{ng serve --open}
    \item if you want to change the default port(4200), do it with the command : \textit{ng serve --open --port 5500}
    \\5500 is the new port
    \item Setting up JSON Server database :
    \begin{enumerate}
        \item create json file database : 
        \\ \textit{touch db.json}
        \item launch the server : 
        \\ make json server extension to watch the file : 
        \\ json-server watch db.json --port 5000
    \end{enumerate}
    
\end{itemize}
\section{HTML Form validation}
\begin{itemize}
    \item required
    \item minlength
    \item maxlength
    \item use the model ngModel : to validate each input alone : ngModel.errors
    \\ use variables in HTML with : 
   
    \begin{lstlisting}
       Example :  <input required [(ngModel)]="article.title"  #mytitle = "[(ngModel)]">
        \end{lstlisting}
    \item ngForm : to validate the whole form, if only one input is invalid, the ngForm.invalid will be true
    \item use ngClass to change the class of HTML elements
        Example : 
        display is a boolean variable in the componenent TypeScript class
        \begin{lstlisting}
        ngClass = "{
            'btn-success' : !display,
            'btn-dark'  : display
        
        }"
        \end{lstlisting}
        
        
\end{itemize}

\section{installing components and services}
\begin{itemize}
    \item generate new component : Run this command \\ \textit{ng g c name\_component\_here}
    \item generate new service : Run this command \\ \textit{ng g s name\_service\_here}
    \item 
    \begin{verbatim}
            ng g c components/navbar --spec=false

    \end{verbatim}

\end{itemize}

\section{Some Javascript and TypeScript tips}
\begin{itemize}
    \item \textit{var} : declaring variables on global scope
    \item \textit{let} : declaring variables on \textbf{current scope only}
    \item \textit{const} : declaring constants
    \item Templates Strings : Concatenate strings using \$
    \\ \textbf{Example : } 
    \begin{lstlisting}  
        console.log('Salam \$\{variable\}')
    \end{lstlisting}
    \item anonymous functions
    \item arrow functions
    
    \item Callbacks : provides asynchronous programming.
    \item delete element from list : 
     \begin{lstlisting}  
        array.splice(indexofElement)
    \end{lstlisting}
   
    \item !! : checking for emptiness 
    \\ \textbf{Example : } 
   
    \begin{lstlisting}  
        if(!!string_here)
    \end{lstlisting}
    will check if string\_here is empty and return true, false otherwise
    
\end{itemize}
\section{Some Google Cloud Products Tips}
\begin{itemize}
    \item Firebase vs FireStorage :
    \begin{itemize}
        \item Firebase is a plateform : many products
        \item one product is \textit{FireStore} database
        \item FireAuth is a product of FireBase that provides security for FireStore
    \end{itemize}
\end{itemize}

\section{Some Bootstrap Tips}
\begin{itemize}
    \item Sizes : 
    \begin{itemize}
        \item lg : Large screens : TV
        \item md : Medium screens : PCs
        \item sm : Small screens : tablets
    \end{itemize}
    \item bootswatch : a superset of bootstrap with ready themes
\end{itemize}

\section{Some good visual studio extensions}
\begin{itemize}
    \item Bracket Pair Colorizer : colorising brackets in code
    \item Bootstrap 4 snippets : provides Bootstrap ready code snippets directly in Visual Studio Code
    \item Auto Import
\end{itemize}
\section{Some nice npm packages}
\begin{itemize}
    \item sweetalert2
    \item emmet
    \item json-server install it with \textit{npm install json-server}
    
\end{itemize}

\section{Some Angular tips}
\begin{itemize}
    \item ng-template
    \item router-outler
    \\Example : \begin{verbatim}
        <router-outlet> </router-outlet>
    \end{verbatim}
    Angular will look for the proper route in app.module.ts stored in array variable \textit{routes} and 
    
    \item Configure bootstrap : angular.json and build/scripts and copy path to CSS from node\_modules package
    \item Angular provides SPA, so while developing, pages should not refresh 
    \item filtering with pipes is supported 
    \begin{verbatim}
        {{lastName | capitalize}} will capitalize the variable lastName
        \\Some pipes : uppercase, lowercase, currency, date,...etc 
    \end{verbatim}
    \item FormModule : for form driven *ngForm
    \item ReactiveFormsModule : reactive forms 
\end{itemize}
\section{Good ressources}
\begin{itemize}
    \item Npm packages : \url{https://www.npmjs.org}
    \item Open source projects : \url{www.github.com}
    \item Regex validation : \url{https://www.regex101.com}
    \item Icons and fonts : \url{https://fontawesome.com/v4.7.0/}
    \item JSON viewer, a chrome pluggin for rendering pretty JSON data : \url{https://chrome.google.com/webstore/detail/json-viewer/gbmdgpbipfallnflgajpaliibnhdgobh}
    \item Fake online REST API for demos :  \url{https://jsonplaceholder.typicode.com/}
    \item Mr Mohammed Idbrahim Youtube Channel  : \url{https://www.youtube.com/channel/UCAeBi3qwZ9pj52yYHm3WEZQ}
    \item Mr Mohammed Idbrahim website : \url{https://brightcoding.teachable.com/}
    \item UIKit \url{www.getuikit.com} 
    \item Discord \url{www.discordapp.com} for efficient team communication
\end{itemize}

\end{document}
